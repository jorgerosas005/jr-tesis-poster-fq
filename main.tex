\documentclass[final]{beamer}
%% Possible paper sizes: a0, a0b, a1, a2, a3, a4.
%% Possible orientations: portrait, landscape
%% Font sizes can be changed using the scale option.
\usepackage[size=a0,orientation=portrait,scale=1.1]{beamerposter}

\usetheme{gemini}
\usecolortheme{smu}
\useinnertheme{rectangles}

% ====================
% Packages
% ====================

\usepackage[utf8]{inputenc}
\usepackage{graphicx}
\usepackage{booktabs}
\usepackage{tikz}
\usepackage{pgfplots}

% =====================
% Packages added by me
% =====================

\usepackage{chemformula} % para fórmulas químicas
\usepackage{mhchem}      % alternativa para ecuaciones químicas
\usepackage[backend=biber,style=numeric]{biblatex}
\usepackage{float}

% ====================
% Lengths
% ====================

% If you have N columns, choose \sepwidth and \colwidth such that
% (N+1)*\sepwidth + N*\colwidth = \paperwidth
\newlength{\sepwidth}
\newlength{\colwidth}
\setlength{\sepwidth}{0.03\paperwidth}
\setlength{\colwidth}{0.45\paperwidth}

\newcommand{\separatorcolumn}{\begin{column}{\sepwidth}\end{column}}

% ====================
% Logo (optional)
% ====================

% LaTeX logo taken from https://commons.wikimedia.org/wiki/File:LaTeX_logo.svg
% use this to include logos on the left and/or right side of the header:
\logoright{\includegraphics[height=8cm]{logos/logo-right.png}}
\logoleft{\includegraphics[height=8cm]{logos/logo-left.png}}

% ====================
% Footer (optional)
% ====================

\footercontent{
	ABC Conference 2025, Tokyo, Japan \hfill
	\insertdate \hfill
	\href{mailto:myemail@exampl.com}{\texttt{myemail@example.com}}
}
% (can be left out to remove footer)

% ====================
% My own customization
% - BibLaTeX
% - Boxes with tcolorbox
% - User-defined commands
% ====================
\input{custom-defs.tex}

%% Reference Sources
\addbibresource{referencias.bib}
\renewcommand{\pgfuseimage}[1]{\includegraphics[scale=2.0]{#1}}

\title{Estudio de la solvatación del ion \ce{Cu^{2+}} en medios polares mediante dinámica molecular con DFT/M06-2X}

\author{Jorge Angel Rosas Martínez \inst{1} \and César Iván León Pimentel \inst{2}}
\institute[shortinst]{\inst{1} Facultad de Química, UNAM }

\date{Agosto 2025}

\begin{document}
	
\begin{frame}[t]
	
	\begin{columns}[t]
		\separatorcolumn

		\begin{column}{\colwidth}
			\begin{block}{Introducción}
				El ion \ce{Cu^{2+}} desempeña un papel fundamental en reacciones de oxidación-reducción, transporte electrónico y formación de complejos metaloproteicos, procesos estrechamente ligados a funciones fisiológicas clave y a enfermedades neurodegenerativas como el Alzheimer y el Parkinson \cite{Me-2022-01}. Tiene aplicaciones en el diseño de fármacos, catálisis homogénea, sensorización iónica, y en el análisis molecular de sistemas biológicos \cite{Me-2023-01}. En este estudio, se analizaron los sistemas \ce{[Cu(H2O)_{40}]^{2+}} y \ce{[Cu(CH3OH)_{40}]^{2+}}, con el objetivo de hacer un análisis estructural y energético mediante simulaciones de dinámica molecular a 300 K durante 22.5 ps.

			\end{block}

			
		\end{column}

		\separatorcolumn

		\begin{column}{\colwidth}
			\begin{block}{Metodología}
				Las simulaciones de dinámica molecular de Born-Oppenheimer se llevaron a cabo en el programa ORCA v6.1 \cite{orca6.1}, utilizando condiciones de ensamble NVT y control de temperatura mediante un termostato de Nosé-Hoover en cadena. Las velocidades iniciales se generaron conforme a la distribución de Maxwell-Boltzmann a 300 K.

				ORCA integra el movimiento atómico empleando el algoritmo de Verlet de velocidad, y en cada paso temporal resuelve la ecuación de Schrödinger independiente del tiempo mediante el método de campo autoconsistente y teoría de los fucionales de la densidad (SCF-DFT por sus siglas en inglés), utilizando el funcional M06-2X y el nivel de teoría 6-31G$^\ast$. Este procedimiento permite obtener las fuerzas necesarias para la evolución del sistema conforme a las ecuaciones de movimiento de Newton.
			\end{block}			
		\end{column}
		
		\separatorcolumn
	
	\end{columns}

	\begin{columns}[t]
	
		\begin{column}{2\colwidth+\sepwidth}
			\begin{alertblock}{Introducción}

				Las siguientes figuras son representativas:

				\begin{figure}[H]
					\centering
					\begin{minipage}[b]{0.25\textwidth}
						\centering
						\includegraphics[width=\textwidth]{logos/Cu-10CH4O.png}
					\end{minipage}%
					\hfill
					\begin{minipage}[b]{0.25\textwidth}
						\centering
						\includegraphics[width=\textwidth]{logos/Cu-30CH4O.png}
					\end{minipage}
					\hfill
					\begin{minipage}[b]{0.25\textwidth}
						\centering
						\includegraphics[width=\textwidth]{logos/Cu-40CH4O.png}
					\end{minipage}

					\caption{Energías de enlace por molécula de solvente calculada para las estructuras de solvatación óptimas de \ce{[Cu(H2O)_{n}]^{2+}} (izquierda) y \ce{[Cu(CH3OH)_{n}]^{2+}} (derecha), con $n = 1,\ 2,\ \ldots,\ 10,\ 30,\ 40$, mediante cálculos de optimización con los niveles de teoría 6-31G$^\ast$, 6-31+G$^\ast$ y 6-31++G$^{\ast\ast}$. Se concluye que la base 6-31G$^\ast$ ofrece un buen equilibrio entre precisión y costo computacional. Gráficas pendientes}
					\label{fig:bases}
				\end{figure}	

				\begin{figure}[H]
					\centering
					\begin{minipage}[b]{0.25\textwidth}
						\centering
						\includegraphics[width=\textwidth]{logos/Cu-10H2O.png}
					\end{minipage}%
					\hfill
					\begin{minipage}[b]{0.25\textwidth}
						\centering
						\includegraphics[width=\textwidth]{logos/Cu-30H2O.png}
					\end{minipage}
					\hfill
					\begin{minipage}[b]{0.25\textwidth}
						\centering
						\includegraphics[width=\textwidth]{logos/Cu-40H2O.png}
					\end{minipage}

					\caption{Energías de enlace por molécula de solvente calculada para las estructuras de solvatación óptimas de \ce{[Cu(H2O)_{n}]^{2+}} (izquierda) y \ce{[Cu(CH3OH)_{n}]^{2+}} (derecha), con $n = 1,\ 2,\ \ldots,\ 10,\ 30,\ 40$, mediante cálculos de optimización con los niveles de teoría 6-31G$^\ast$, 6-31+G$^\ast$ y 6-31++G$^{\ast\ast}$. Se concluye que la base 6-31G$^\ast$ ofrece un buen equilibrio entre precisión y costo computacional. Gráficas pendientes}
					\label{fig:bases}
				\end{figure}


			\end{alertblock}
		\end{column}

	\end{columns}

	\begin{columns}[t]

		\separatorcolumn
		
		\begin{column}{\colwidth}
			
			\begin{block}{An example block containing some math}{}

				%%%
				\begin{figure}[H]
					\centering
					\begin{minipage}[b]{0.48\textwidth}
						\centering
						\includegraphics[width=\textwidth]{logos/bases_agua.png}
					\end{minipage}%
					\hfill
					\begin{minipage}[b]{0.48\textwidth}
						\centering
						\includegraphics[width=\textwidth]{logos/bases_metanol.png}
					\end{minipage}

					\caption{Energías de enlace por molécula de solvente calculada para las estructuras de solvatación óptimas de \ce{[Cu(H2O)_{n}]^{2+}} (izquierda) y \ce{[Cu(CH3OH)_{n}]^{2+}} (derecha), con $n = 1,\ 2,\ \ldots,\ 10,\ 30,\ 40$, mediante cálculos de optimización con los niveles de teoría 6-31G$^\ast$, 6-31+G$^\ast$ y 6-31++G$^{\ast\ast}$. Se concluye que la base 6-31G$^\ast$ ofrece un buen equilibrio entre precisión y costo computacional. Gráficas pendientes}
					\label{fig:bases}
				\end{figure}
				
			\end{block}
			
			\begin{block}{An example block containing some math}{}
				
				\begin{figure}[H]
					\centering
					% Primera imagen
					\begin{minipage}[c]{0.49\textwidth} % Adjust width as needed, ensure total < 1.0\textwidth
						\centering
						\includegraphics[width=\textwidth]{logos/RDF-OUTPUT-HISTOGRAMA.png}
						\caption{Función de distribución radial para el ion \ce{Cu^{2+}} en metanol (\ce{CH_{4}O}) y agua (\ce{H_{2}O}).}
						\label{fig:rdfcu40ch4o}
					\end{minipage}% <--- Important: No space here!
					\hfill % Adds flexible space between minipages, or remove if you want them tight
					% Segunda imagen
					\begin{minipage}[c]{0.49\textwidth} % Adjust width as needed
						\centering
						\includegraphics[width=\textwidth]{logos/ADF-OUTPUT-HISTOGRAMA.png}
						\caption{Función de distribución angular con $r = 2.9$ para el ion \ce{Cu^{2+}} en metanol (\ce{CH_{4}O}) y agua (\ce{H_{2}O}).}
						\label{fig:adfcu40ch4o}
					\end{minipage}
				\end{figure}

			\end{block}		

		\end{column}
	
		\separatorcolumn
		
		\begin{column}{\colwidth}

			\begin{exampleblock}{Resultados y discusión}
				Los resultados obtenidos muestran que la energía de enlace por molécula de solvente para \ce{[Cu(H2O)_{n}]^{2+}} y \ce{[Cu(CH3OH)_{n}]^{2+}} disminuye conforme aumenta el número de moléculas de solvente, lo que indica una mayor estabilidad del ion en medios polares. La función de distribución radial (RDF) y la función de distribución angular (ADF) revelan la estructura local del ion \ce{Cu^{2+}} en ambos solventes, mostrando una preferencia por ciertas distancias y ángulos entre las moléculas de agua y metanol.

				Los resultados sugieren que el ion \ce{Cu^{2+}} forma complejos más estables en metanol que en agua, lo cual es consistente con estudios previos sobre la solvatación de iones metálicos en medios polares.

				
			\end{exampleblock}					

			
			\begin{block}{References}
				\printbibliography[heading=none]
			\end{block}
		\end{column}
		
		\separatorcolumn
	\end{columns}
\end{frame}
\end{document}