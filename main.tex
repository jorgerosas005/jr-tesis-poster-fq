\documentclass[final]{beamer}
%% Possible paper sizes: a0, a0b, a1, a2, a3, a4.
%% Possible orientations: portrait, landscape
%% Font sizes can be changed using the scale option.
\usepackage[size=a0,orientation=portrait,scale=1.1]{beamerposter}

\usetheme{gemini-modificado}
\usecolortheme{smu}
\useinnertheme{rectangles}

% ====================
% Packages
% ====================

\usepackage[utf8]{inputenc}
\usepackage{graphicx}
\usepackage{booktabs}
\usepackage{tikz}
\usepackage{pgfplots}
\pgfplotsset{compat=1.18} % Este lo puse yo
% =====================
% Packages added by me
% =====================

\usepackage{chemformula} % para fórmulas químicas
\usepackage[version=4]{mhchem}      % alternativa para ecuaciones químicas
\usepackage[backend=biber,style=numeric, sorting=none, doi=false, url=false]{biblatex}
\usepackage{float}
\usepackage{subcaption}
%\captionsetup{font=footnotesize} % You can use small, footnotesize, etc.
\usepackage{amsmath}
\usetikzlibrary{shapes.geometric, arrows.meta, positioning}
\usepackage{ragged2e} %Justificar texto en bloques

\AtEveryBibitem{%
  \ifentrytype{article}{%
    \clearfield{title}%
  }{}%
}

% ====================
% Lengths
% ====================

% If you have N columns, choose \sepwidth and \colwidth such that
% (N+1)*\sepwidth + N*\colwidth = \paperwidth
\newlength{\sepwidth}
\newlength{\colwidth}
\setlength{\sepwidth}{0.03\paperwidth}
\setlength{\colwidth}{0.45\paperwidth}

\newcommand{\separatorcolumn}{\begin{column}{\sepwidth}\end{column}}

% ====================
% Logo (optional)
% ====================

% LaTeX logo taken from https://commons.wikimedia.org/wiki/File:LaTeX_logo.svg
% use this to include logos on the left and/or right side of the header:
\logoright{ \includegraphics[height=7.5cm]{logos/Ciencias-Químicas_color.png}}
\logoleft{  \includegraphics[height=11cm]{logos/logo-fq-negro.jpg}}

% ====================
% Footer (optional)
% ====================

\footercontent{
    %\hfill % Mantenemos esto para empujar todo el bloque a la derecha
    \begin{minipage}[t]{0.9\textwidth} % Hacemos la caja un poco más ancha para las 3 columnas
        \usebeamerfont{footline}
		\raggedleft
        
        % Tabla con 3 columnas, separadas por una línea vertical
        \begin{tabular}{l | l |l}
            % Fila 1
            \bfseries C O N T A C T O & % Columna 1
            \texttt{jorgerosas005@gmail.com} & % Columna 2
            \texttt{cesar.leon@quimica.unam.mx} \\[0.5ex] % Columna 3
            
        \end{tabular}
    \end{minipage}
}
% (can be left out to remove footer)

% ====================
% My own customization
% - BibLaTeX
% - Boxes with tcolorbox
% - User-defined commands
% ====================
\input{custom-defs.tex}

%% Reference Sources
\addbibresource{referencias.bib}
\renewcommand{\pgfuseimage}[1]{\includegraphics[scale=2.0]{#1}}

\title{Solvatación del ion \ce{Cu^{2+}} en medios polares mediante dinámica molecular ab initio con DFT/M06-2X}

\author{Jorge Angel Rosas Martínez \inst{1} \and César Iván León Pimentel \inst{2}}

\institute[shortinst]{\inst{1} Facultad de Química, UNAM \inst{2} Depto. de Matemáticas, Facultad de Química, UNAM}
\date{Agosto 2025}

\begin{document}
	
\begin{frame}[t]

	
	\begin{columns}[t]
		\separatorcolumn

		\begin{column}{\colwidth}
			\begin{block}{I N T R O D U C C I Ó N}

				\justifying

				La solvatación del ion cobre (II) \ce{Cu^{2+}} es fundamental para una vasta gama de procesos químicos y biológicos, desde su papel crítico en las actividades enzimáticas, el transporte de oxígeno y la transferencia de electrones \cite{Wa-2023-01, Wa-2024-03} hasta su implicación en trastornos neurodegenerativos \cite{Cu-2014-02}. Una característica central que gobierna su química en fase de solución es la distorsión de Jahn-Teller, consecuencia directa de su configuración electrónica $d^9$ \cite{Cu-2019-01}, la cual imparte una considerable labilidad estructural y una dinámica compleja a su primera capa de solvatación.\\ Este fenómeno ha sido estudiado extensamente en disolución acuosa, generando un vasto cuerpo de conocimiento, aunque no exento de debate sobre la estructura y dinámica de su primera esfera de coordinación. Existe un consenso emergente que apunta a una coexistencia dinámica de especies con números de coordinación 5 y 6 \cite{Wa-2023-01}. Sin embargo, la información disponible para la solvatación de este ion en metanol es considerablemente más escasa y, en ocasiones, contradictoria, con estudios experimentales que reportan números de coordinación promedio que varían desde 4 hasta 6 \cite{Me-2025-01}. \\ Para abordar esta ambigüedad en la información de la literatura se realizó un estudio de Dinámica Molecular Ab initio basada en la Teoría de los Funcionales de la Densidad utilizando M06-2X. Realizamos simulaciones de nanogotas solvatadas de \ce{[Cu(H2O)_{40}]^{2+}} y \ce{[Cu(CH3OH)_{40}]^{2+}} con objeto de contrastar nuestros resultados obtenidos con el resto de trabajos similares con agua y presentar por primera vez en la literatura un estudio riguroso de esta naturaleza para caracterizar la solvatación del \ce{Cu^{2+}} en metanol.


			\end{block}

		\end{column}

		\separatorcolumn

		\begin{column}{\colwidth}

			\begin{alertblock}{C L U S T E R S}

				\begin{figure}[H]
					\centering
					% Fila superior: Complejos con Metanol
					\begin{subfigure}[b]{0.27\textwidth}
						\centering
						\includegraphics[width=\textwidth]{logos/Cu-10CH4O.png}
						\caption{\ce{[Cu(CH3OH)_{10}]^{2+}}}
						\label{fig:cu-10ch4o}
					\end{subfigure}%
					\hfill
					\begin{subfigure}[b]{0.27\textwidth}
						\centering
						\includegraphics[width=\textwidth]{logos/Cu-30CH4O.png}
						\caption{\ce{[Cu(CH3OH)_{30}]^{2+}}}
						\label{fig:cu-30ch4o}
					\end{subfigure}%
					\hfill
					\begin{subfigure}[b]{0.27\textwidth}
						\centering
						\includegraphics[width=\textwidth]{logos/Cu-40CH4O.png}
						\caption{\ce{[Cu(CH3OH)_{40}]^{2+}}}
						\label{fig:cu-40ch4o}
					\end{subfigure}
					
					\vspace{1em} % Espacio vertical entre las filas de imágenes
					
					% Fila inferior: Complejos con Agua
					\begin{subfigure}[b]{0.27\textwidth}
						\centering
						\includegraphics[width=\textwidth]{logos/Cu-10H2O.png}
						\caption{\ce{[Cu(H2O)_{10}]^{2+}}}
						\label{fig:cu-10h2o}
					\end{subfigure}%
					\hfill
					\begin{subfigure}[b]{0.27\textwidth}
						\centering
						\includegraphics[width=\textwidth]{logos/Cu-30H2O.png}
						\caption{\ce{[Cu(H2O)_{30}]^{2+}}}
						\label{fig:cu-30h2o}
					\end{subfigure}%
					\hfill
					\begin{subfigure}[b]{0.27\textwidth}
						\centering
						\includegraphics[width=\textwidth]{logos/Cu-40H2O.png}
						\caption{\ce{[Cu(H2O)_{40}]^{2+}}}
						\label{fig:cu-40h2o}
					\end{subfigure}
					
					\caption{Estructuras de solvatación óptimas para los sistemas \ce{[Cu(CH3OH)_{n}]^{2+}} (fila superior) y \ce{[Cu(H2O)_{n}]^{2+}} (fila inferior), con $n = 10,\ 30,\ 40$.}
					\label{fig:estructuras_solvatacion}
				\end{figure}				

			\end{alertblock}

		\end{column}
		
		\separatorcolumn
	
	\end{columns}

	\begin{columns}[t]
	
		\begin{column}{2\colwidth+\sepwidth}
			\begin{exampleblock}{M E T O D O L O G Í A}{}
				
				\begin{figure}[H]
					\centering
					\begin{tikzpicture}[font=\footnotesize\sffamily,
						node distance = 8mm and 15mm, % Distancia vertical y horizontal
						block/.style = {rectangle, draw, text centered, rounded corners,
										minimum height=2.2cm, align=center},
						diamond_dec/.style = {diamond, draw, text centered, aspect=2,
											inner sep=1pt, text width=2.5cm},
						line/.style = {-{Stealth[length=2.5mm]}}
						]

						% --- Fila Superior ---
						\node[block] (cond) {\textbf{Inicialización}  \\ $R^{N}(t=0)$};
						\node[block, right=of cond] (opt_structure) {\textbf{Optimizar estructura} \\ {\text{Minimizar energía}} \\ $R^{N}_{min}(0)$};
						\node[block, right=of opt_structure](init_vel) {\textbf{Velocidades} \\ \textbf{iniciales} \\$V^{N}(t=0)$};
						\node[block, right=of init_vel] (t_step) { \textbf{Bucle SCF} \\ $t^{(n)}$};



						% --- Fila Intermedia (Bucle SCF) ---
						\node[block, right=3cm of t_step] (propose_rho){ \textbf{Proponer} \\$\rho^{(n)}(r)$};
						\node[block, below=2.3cm of propose_rho] (build_h) {\textbf{Construir hamiltoniano} \\ $H^{(n)}_{KS} = -\frac{1}{2} \nabla^2 + V_{\text{ext}} + V_H^{(n)} + V_{xc}^{(n)}$};
						\node[block, right=of build_h] (solve_ks) {\textbf{Resolver ecuaciones} \\ \textbf{de Kohn-Sham} \\ $H^{(n)}_{KS} \psi_i = \epsilon_i\psi_i$};
						\node[block, right=of solve_ks] (calc_rho) {\textbf{Calcular nueva densidad} \\ $\rho^{(n+1)}(r) = \sum_{i=1}^{N_{occ}} |\psi_i(r)|^2$};
						\node[diamond_dec, above=1.8cm of calc_rho, xshift=0cm, minimum width=6cm, minimum height=2.5cm] (scf_check) {  \textbf{¿Converge?}  };
						\node[block, right=of scf_check, xshift=1cm] (forces) {\textbf{Calcular energía y fuerzas} \\ $E[\rho] = T_s[\rho] + V_H[\rho] + E_{xc}[\rho] + V_{ext}[\rho]$ \\ $F_I = -\nabla_{R_I} E[\rho]$};

						% --- Fila Inferior (Bucle de Dinámica) ---
						% ***** LÍNEA MODIFICADA AQUÍ *****
						\node[block, below=5.5cm of forces] (vel_update) {\textbf{Actualizar velocidades} \\ \text{Integración + Nosé-Hoover} \\  $V^{N}(t + \Delta t)$ , $T \approx 300\,\text{K}$ };
						\node[block, left=11.5cm of vel_update] (pos_update) {\textbf{Actualizar posiciones} \\ \text{Integración} \\ $R^{N}( t + \Delta t)$};

						% --- Inicio y Fin ---
						\node[block, left=of cond] (start) {Inicio};
						\node[diamond_dec, left=11.5cm of pos_update] (main_loop_check) {¿$t < 20\,\text{ps}$?};
						\node[block, left=of main_loop_check] (analysis) {\textbf{Análisis}};
						\node[block, left=of analysis] (end) {Fin};

						% --- Conexiones con Flechas ---
						% Flujo principal
						\draw[line] (start) -- (cond);
						\draw[line] (cond) -- (opt_structure);
						\draw[line] (opt_structure) -- (init_vel);
						\draw[line] (init_vel) -- (t_step);
						\draw[line] (t_step) -- (propose_rho);

						% Bucle SCF
						\draw[line] (propose_rho) -- (build_h);
						\draw[line] (build_h) -- (solve_ks);
						\draw[line] (solve_ks) -- (calc_rho);
						\draw[line] (calc_rho) -- (scf_check);
						\draw[line] (scf_check) -- node[above] {Sí} (forces);
						\draw[line] (scf_check.west) -- ++(-0.5,0) -| node[below, pos=0.25] {$\rho^{(n)}(r) = \rho^{(n+1)}(r)$}  node[above, pos=0.25] {No} (propose_rho.east);

						% Bucle de Dinámica
						\draw[line] (forces) -- (vel_update);
						\draw[line] (vel_update) -- (pos_update);
						\draw[line] (pos_update) -- (main_loop_check);

						% Bucle principal y final
						\draw[line] (main_loop_check.north) -| node[left, pos=0.8] {Si} (t_step.south);
						\draw[line] (main_loop_check) -- node[below, pos=0.5] {No} (analysis);
						\draw[line] (analysis) -- (end);

						

					\end{tikzpicture}
					\caption{Diagrama general de cálculo.}
					\label{fig:metodologia}
				\end{figure}

				%Para la fase de inicialización de $R^{N}(t=0)$ se utilizó Packmol v2, se hicieron una serie de optimizaciones para estructuras desde$n = 1,\ 2,\ \ldots,\ 10,\ 30,\ 40$ para agua y metanol (ver figura 3) y determinar la elección de base 6-31G*. Los parámetros de la simulación fueron los siguientes:

				%\begin{itemize}
					%\item Funcional intercambio y correlación M06-2X con 6-31G*
					%\item Las velocidades iniciales se calculan conforme a la distribución de Maxwell-Boltzmann a $T \approx 300\,\text{K}$
					%\item La duración de la DM fue 22.5 ps (termalización 2.5 ps) con $\Delta t = 0.5\,\text{fm}$ (45,000 ciclos de cálculo en total)
					%\item Ensamble NVT con termostato Nosé-Hoover en cadena (4) $T \approx 300\,\text{K}$ e integrador de orden superior Yoshida 7
				%\end{itemize}

				%Los cálculos de optimización y las dinámicas moleculares se realizaron en Orca v6.1 \cite{orca}.

			\end{exampleblock}		

		\end{column}

	\end{columns}

	\begin{columns}[t]

		\separatorcolumn
		
		\begin{column}{\colwidth}
			
			\begin{block}{E N E R G Í A}{}

				%%%
				\begin{figure}[H]
					\centering
					\begin{minipage}[b]{0.48\textwidth}
						\centering
						\includegraphics[width=\textwidth]{logos/bases_agua.png}
					\end{minipage}%
					\hfill
					\begin{minipage}[b]{0.48\textwidth}
						\centering
						\includegraphics[width=\textwidth]{logos/bases_metanol.png}
					\end{minipage}

					\caption{Energías de enlace por molécula de solvente calculada para las estructuras de solvatación óptimas de \ce{[Cu(H2O)_{n}]^{2+}} (izquierda) y \ce{[Cu(CH3OH)_{n}]^{2+}} (derecha), con $n = 1,\ 2,\ \ldots,\ 10,\ 30,\ 40$, mediante cálculos de optimización con los niveles de teoría 6-31G$^\ast$, 6-31+G$^\ast$ y 6-31++G$^{\ast\ast}$. Resultados obtenidos con MP2 \cite{Me-2022-01} y validación de M06-2X respecto a MP2 \cite{Me-2023-01}.}
					\label{fig:bases}
				\end{figure}
				
			\end{block}
			
			\begin{exampleblock}{C O N C L U S I O N E S}{}
				Conclusiones preeliminares:
				. \\
				. \\
				. \\
				. \\
				. \\
				. \\
				. \\
				. \\
				. \\
				. \\
				. \\
				. \\
				. \\
				. \\
				. \\
				. \\			
				. \\
				. \\


			\end{exampleblock}		

		\end{column}
	
		\separatorcolumn
		
		\begin{column}{\colwidth}
		
			\begin{block}{C A R A C T E R I Z A C I Ó N }{}
				
				\begin{figure}[H]
					\centering
					% Primera imagen
					\begin{minipage}[c]{0.49\textwidth} % Adjust width as needed, ensure total < 1.0\textwidth
						\centering
						\includegraphics[width=\textwidth]{logos/RDF-OUTPUT-HISTOGRAMA.png}
						\caption{Función de distribución radial para el ion \ce{Cu^{2+}} en metanol (\ce{CH_{4}O}) y agua (\ce{H_{2}O}).}
						\label{fig:rdfcu40ch4o}
					\end{minipage}% <--- Important: No space here!
					\hfill % Adds flexible space between minipages, or remove if you want them tight
					% Segunda imagen
					\begin{minipage}[c]{0.49\textwidth} % Adjust width as needed
						\centering
						\includegraphics[width=\textwidth]{logos/ADF-OUTPUT-HISTOGRAMA.png}
						\caption{Función de distribución angular con $r = 2.9$ para el ion \ce{Cu^{2+}} en metanol (\ce{CH_{4}O}) y agua (\ce{H_{2}O}).}
						\label{fig:adfcu40ch4o}
					\end{minipage}
				\end{figure}

			\end{block}	

			\begin{block}{Bibliografía}
				\printbibliography[heading=none]
			\end{block}

			\begin{thm}{Agradecimientos}{}
				
				Especiales agradecimientos a los siguientes programas por brindar los recursos y financiación:
				\begin{itemize}
					\item Proyecto de supercómputo \textbf{LANCAD-UNAM-DGTIC-435}.
					\item Programa \textbf{UNAM-PAPIIT IA201724}.
				\end{itemize}
								
			\end{thm}

		\end{column}
		
		\separatorcolumn
	\end{columns}
\end{frame}
\end{document}